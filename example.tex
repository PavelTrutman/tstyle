\documentclass[mathserif,aspectratio=169]{beamer}
\usepackage[english]{babel}
\usepackage[utf8]{inputenc}
\title{There Is No Largest Prime Number}
\subtitle{The Famous Proof}
\date[CVPR 2018]{27th International Symposium of Prime Numbers}
\date[CVPR 2018]{\today}
\author[Pavel Trutman]{Pavel Trutman \\\texttt{pavel.trutman@cvut.cz}}
\institute[CIIRC, CTU]{Czech Institute of Informatics, Robotics, and Cybernetics\\ Czech Technical University in Prague}

\usepackage{tstyle/beamerthemetstyle}

\begin{document}

\begin{frame}
  \titlepage
\end{frame}

\section{Intoduction}
\subsection{Sub introduction}
\begin{frame} 
  \frametitle{There Is No Largest Prime Number} 
  \framesubtitle{The proof uses \textit{reductio ad absurdum}.} 
  \begin{theorem}
  There is no largest prime number. \end{theorem} 
  \begin{enumerate} 
  \item<1-| alert@1> Suppose $p$ were the largest prime number. 
  \item<2-> Let $q$ be the product of the first $p$ numbers. 
  \item<3-> Then $q+1$ is not divisible by any of them. 
  \item<1-> But $q + 1$ is greater than $1$, thus divisible by some prime
  number not in the first $p$ numbers.
  \end{enumerate}
\end{frame}

\section{Conclusions}
\begin{frame}{A longer title}
\begin{itemize}
\item one
\item two
\end{itemize}
\end{frame}

\end{document}
